
\documentclass{ctexart}

\usepackage{graphicx}

\usepackage{amsmath}

\usepackage{geometry}
\geometry{left=2cm,right=2cm}

\title{作业二Shell中函数的编写与运用}

\author{南子谦 \\ 信息与计算科学\quad 3210104676}

\begin{document}

\maketitle

\section{引入}
在书\textit{Beginning Linux Programming}中, 我们可以发现Shell作为脚本语言,仍然拥有大多数高级语言的功能;如同C++,Shell也能够编写函数。针对书P48中的例子,我获得了以下收获。

\section{说明}
书中\verb!yes_or_no!函数兼具了向屏幕输出内容和传输数值的功能, 其函数如下:
\begin{verbatim}
    yes_or_no()
    {
        echo "Is your name $*?"
        while true
        do
            echo -n "Enter yes or no: "
            read x
            case "$x" in
                y | yes ) return 0;;
                n | no ) return 1;;
                * ) echo "Answer yes or no!"
            esac
        done
    }
\end{verbatim}

\par 然后其主程序如下:

\begin{verbatim}
    echo "Original parameters are $*"
    if yes_or_no "$1"
    then
        echo "Hi $1, nice name"
    else
        echo "Never mind"
    fi
    exit 0
\end{verbatim}

\par 其输出如下:

\begin{verbatim}
    $ **./my_name Rick Neil**
    Original parameters are Rick Neil
    Is your name Rick ?
    Enter yes or no: **yes**
    Hi Rick, nice name
    $
\end{verbatim}

\section{我的理解}
在程序开始执行时, \verb!yes_or_no!被定义了,在分支语句\verb!if!中, 上述函数被执行, 且环境变量\verb!$1!传入函数. 函数接收传入的参数, 依据用户的输入给出返回值. 

\section{总结}
经过下午的研究, 我更加深入地了解了Shell语言, 并与以前学过的一些语言建立了联系. Shell语言的case就像是C语言的switch case,它们的返回操作
同样是return.


\bibliographystyle{plain}
\bibliography{test}

\cite{abcd2022}

\end{document}
