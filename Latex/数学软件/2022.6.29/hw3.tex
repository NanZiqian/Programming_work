
\documentclass{ctexart}

\usepackage{graphicx}

\usepackage{amsmath}


\title{作业三:我的Linux工作环境}

\author{南子谦 \\ 信息与计算科学\quad 3210104676}

\begin{document}

\maketitle

\section{我的Linux}
发行版名称:Ubuntu
\par 版本号:Ubuntu 22.04 LTS
\par 安装的软件:Synaptic:用于管理已装软件与安装新软件;
\\ Emacs:文本编辑器
\\ Fcitx:支持 Linux 和 Unix 系统的,支持扩展的输入法框架
\\ Visual Studio Code:微软的文本编辑器
\\ CMake:是一个开源的跨平台工具系列,旨在构建、测试和打包软件;
\\ dx:?
\\ gcc:GNU编译器套件,是由GNU开发的编程语言译器,GNU C在标准C语言的基础上进行了部分方便开发的扩展
\\ texlive:Latex的数据包
\\ trilinos:Trilinos项目是一个由开发人员,用户和用户开发人员组成的社区,专注于在面向对象的软件框架内协作创建算法和支持技术,以解决大规模,复杂的多物理场工程和新兴高性能计算(HPC)架构上的科学问题。
\\ open-vm-tools:方便主机和虚拟机交流的插件

\section{Linux下的未来工作}
接下来的半年中,我会在学习课程\textbf{数据结构与算法}时,在撰写课程论文,练习C语言编程与Shell编程时会使用到Linux环境。
\par 目前我的工作环境较适合我的工作环境,因为我的个人电脑并非全部用来工作,在虚拟机这个相对隔离的环境中工作有利于学习与娱乐分离;并且该环境已经具备了我对软件的要求,能够做到直接使用,效率较高。
\section{数据安全与环境稳定}
关于代码安全:我将使用github进行我的代码管理,这样既能做到多设备的同步也能做到代码的安全;
\par 关于文献安全:我将使用OneDrive作为我的个人网盘,存放个人文件和学习文献;OneDrive同样可以做到释放空间和多设备同步的效果;
\par 关于工作环境的安全:如果我需要对工作环境进行新一步的探索,我将创建新的虚拟机进行探索,因为虚拟机十分的方便。

\bibliographystyle{plain}
\bibliography{hw3}

\nocite{qiu2012}

\end{document}
