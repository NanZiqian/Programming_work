\documentclass{ctexart}

\usepackage{graphicx}
\usepackage{amsmath}

\title{作业二: 在shell中执行代码之AND Lists}


\author{钟谢田 \\ 数学与应用数学3190103159}

\begin{document}

\maketitle
Shell 脚本含有ADD LIST是怎么回事呢?Shell 脚本相信大家都很熟悉,但是Shell 脚本含有ADD LIST是怎么回事呢,下面就让小编带大家一起了解吧。

Shell 脚本含有ADD LIST,其实就是有LIST包含ADD,大家可能会很惊讶Shell 脚本怎么会含有ADD LIST呢?但事实就是这样,小编也感到非常惊讶。

这就是关于Shell 脚本含有ADD LIST的事情了,大家有什么想法呢,欢迎在评论区告诉小编一起讨论哦!
\section{介绍与代码}
有时分支结构需要很多的判断,如果嵌套使用IF的话,不仅不美观,难以维护,而且代码量也很大,十分不利程序员的身心健康。为此,我们可以使用ADD LIST,代码如下:

\begin{verbatim}
    #!/bin/sh

    touch file_one
    rm -f file_two
    rm -f file_drei
    touch file_funf
    if [ -f file_one ] && echo "hello" && [ -f file_two ] && echo " There"
    then 
        echo "in if"
    else
        echo "in else"
    fi
    
    exit 0    
\end{verbatim}

其中\verb![ -f file_one ] && echo "hello" && [ -f file_two ] && echo " There"!就是我们的ADD LIST了。怎么样,是不是很方便呢?

运行\verb|chmod +x first|使之成为可执行的文件。

\section{运行结果与分析改进}
运行\verb|./first|结果如下:

\begin{verbatim}
hello
in else
\end{verbatim}

可以注意到\verb| There|并没有被输出,这是因为逻辑表达式短路,在AND语句中遇到了值为假的语句立即结束运行,返回false。

下面结合OR进行了一些修改,运行\verb|./second|:

\begin{verbatim}
    #!/bin/sh

    touch file_one
    rm -f file_two
    rm -f file_drei
    touch file_funf
    if [ -f file_one ] && echo "hello" && [ -f file_two ] && echo " There" || [ -f file_drei ] || echo " Vier" && [ -f file_funf ]
    then 
        echo "in if"
    else
        echo "in else"
    fi
    
    exit 0    
\end{verbatim}

程序的运行结果是

\begin{verbatim}
hello
 Vier
in if
\end{verbatim}

文件file\_one与file\_funf生成了,其他文件不存在。

符合预期。

\end{document}

